% LaTeX Curriculum Vitae Template
%
% Copyright (C) 2004-2009 Jason Blevins <jrblevin@sdf.lonestar.org>
% http://jblevins.org/projects/cv-template/
%
% You may use this document as a template to create your own CV
% and you may redistribute the source code freely. No attribution is
% required in any resulting documents. I do ask that you please leave
% this notice and the above URL in the source code if you choose to
% redistribute this file.
%
% This document has been modified heavily by Josiah Couch.
%

\documentclass[letterpaper]{article}

\usepackage{hyperref}
\usepackage{geometry}

%Fix spacing after sections
\usepackage[compact]{titlesec}  
\titlespacing{\section}{0pt}{*0}{*0}
\titlespacing{\subsection}{0pt}{*0}{*0}
\titlespacing{\subsubsection}{0pt}{*0}{*0}

% Comment the following lines to use the default Computer Modern font
% instead of the Palatino font provided by the mathpazo package.
% Remove the 'osf' bit if you don't like the old style figures.
\usepackage[T1]{fontenc}
\usepackage[sc,osf]{mathpazo}

% Set your name here
\def\name{Josiah Couch}

% Replace this with a link to your CV if you like, or set it empty
% (as in \def\footerlink{}) to remove the link in the footer:
\def\footerlink{}

% The following metadata will show up in the PDF properties
\hypersetup{
  colorlinks = true,
  urlcolor = black,
  pdfauthor = {\name},
  pdfkeywords = {physics, mathematics, quantum mechanics, quantum information, quantum computing, python},
  pdftitle = {\name: resume},
  pdfsubject = {resume},
  pdfpagemode = UseNone
}

\geometry{
  %body={6.5in, 8.5in},
  left=0.25in,
  right=0.25in,
  top=0.15in,
  bottom=0.25in
}

% Customize page headers
\pagestyle{myheadings}
\markright{\name}
\thispagestyle{empty}

% Custom section fonts
\usepackage{sectsty}
\sectionfont{\rmfamily\mdseries\Large\bf\sectionrule{0ex}{0pt}{-1ex}{1pt}}
\subsectionfont{\rmfamily\mdseries\bf\normalsize}
\subsubsectionfont{\rmfamily\mdseries\itshape\normalsize}

% Other possible font commands include:
% \ttfamily for teletype,
% \sffamily for sans serif,
% \bfseries for bold,
% \scshape for small caps,
% \normalsize, \large, \Large, \LARGE sizes.

% Don't indent paragraphs.
\setlength\parindent{0em}
%Control spacing between lines and after paragraphs
\setlength{\parskip}{0em}
\renewcommand{\baselinestretch}{0}

% Section spacings
\usepackage{titlesec}
\titlespacing*{\section}{0em}{0em}{0em}
\titlespacing*{\subsection}{0em}{0em}{0em}
\titlespacing*{\subsubsection}{0em}{0em}{0em}


% Control lists
\renewenvironment{itemize}{
  \begin{list}{$\bullet$}{
    \setlength{\itemsep}{0em}
    \setlength{\parskip}{0em}
    \setlength{\parsep}{0em} 
    \setlength{\topsep}{0em} 
  }
}{
  \end{list}
}

\begin{document}

% Place name at left
%{\huge \name}

% Alternatively, print name centered:
\centerline{\LARGE \name}

\begin{center}

\small{josiah.couch@utexas.edu $\bullet$ josiahcouch.com}

\end{center}

\section*{Summary}

\begin{itemize}

\item Ph.D. in Theoretical physics (expected spring/summer 2020)

\item Advanced skills in mathematics and quantitative problem-solving

\item Data analysis experience using Python and C/C++

\item Experience doing original research in a collaborative environment

\end{itemize}

\section*{Education}

\subsection*{University of Texas at Austin, Austin, TX}
\subsubsection*{Ph.D in Physics (3.6 GPA) \hfill August 2013 - May 2021 (expected)}

\begin{itemize}
\item Adviser: Willy Fischler
\item Field: Quantum gravity and AdS/CFT
\end{itemize}

\subsection*{Oklahoma State University, Stillwater, Oklahoma}
\subsubsection*{B.S in Physics and B.S. in Mathematics \hfill August 2009 - August 2013}

\section*{Experience}

\subsection*{University of Texas at Austin, Austin, TX} 
\subsubsection*{Ph.D. Candidate \hfill August 2013 - Present}
    \begin{itemize}
        
        \item Solve complex problems related to quantum gravity using mathematics, including differential geometry, linear algebra, and differential equations, along with the Mathematica software package
        \item Apply concepts from quantum information theory and quantum computing in a gravitational context
        \item Collaborate effectively with other researchers, including those from other universities and other subfields of physics
        \item Communicate research findings both to subject matter experts and more general audiences through publications as well as formal and informal talks
        
    \end{itemize}
    
\subsubsection*{Teaching Assistant \hfill August 2013 - Present}
    \begin{itemize}
    
        \item Evaluate student work and provide critical feedback
        \item Model analytic problem solving to students through official solutions to problem sets and in live review sessions
        \item Coach students in quantitative problem solving as well as understanding of course subject matters
        \item Subjects include engineering physics, forensic science, electricity, and magnetism at the undergraduate level, as well as quantum mechanics at both the undergraduate and graduate level
        
    \end{itemize}
    
\subsubsection*{Graduate Research Assistant with Center for Particles and Fields under Prof. Peter Onyisi \hfill June - July 2014}
    \begin{itemize}
    
        \item Optimized feature selection for a particle physics analysis using simulated data, python, and the TMVA boosted decision tree implementation (used throug pyROOT)
        \item Developed a python script to pass configuration file from database to fast tracker system
        
    \end{itemize}

\subsection*{Oklahoma State University, Stillwater, OK}
\subsubsection*{Undergraduate Researcher in Experimental Particle Physics Group under Prof. Flera Rizatdinova \hfill January 2010 - May 2012}
    \begin{itemize}
        \item Analyzed particle physics data using C/C++ and the ROOT package
    \end{itemize}

\section*{Skills}

Quantitative Problem Solving, Research, Collaboration, Advanced Mathematics, Python, Java, Git, Mathematica, LaTex

\section*{Interests}

Quantum Gravity, Quantum Computing, Programming, Machine Learning, Photography, Video Games, Mathematics, Languages

\end{document}
