% LaTeX Curriculum Vitae Template
%
% Copyright (C) 2004-2009 Jason Blevins <jrblevin@sdf.lonestar.org>
% http://jblevins.org/projects/cv-template/
%
% You may use this document as a template to create your own CV
% and you may redistribute the source code freely. No attribution is
% required in any resulting documents. I do ask that you please leave
% this notice and the above URL in the source code if you choose to
% redistribute this file.
%
% This document has been modified heavily by Josiah Couch.
%

\documentclass[letterpaper]{article}

\usepackage{hyperref}
\usepackage{geometry}

%Fix spacing after sections
\usepackage[compact]{titlesec}  
\titlespacing{\section}{0pt}{*0}{*0}
\titlespacing{\subsection}{0pt}{*0}{*0}
\titlespacing{\subsubsection}{0pt}{*0}{*0}

% Comment the following lines to use the default Computer Modern font
% instead of the Palatino font provided by the mathpazo package.
% Remove the 'osf' bit if you don't like the old style figures.
\usepackage[T1]{fontenc}
\usepackage[sc,osf]{mathpazo}

% Set your name here
\def\name{Josiah Couch}

% Replace this with a link to your CV if you like, or set it empty
% (as in \def\footerlink{}) to remove the link in the footer:
\def\footerlink{}

% The following metadata will show up in the PDF properties
\hypersetup{
  colorlinks = true,
  urlcolor = black,
  pdfauthor = {\name},
  pdfkeywords = {physics, mathematics, quantum mechanics, quantum information, quantum computing, python},
  pdftitle = {\name: resume},
  pdfsubject = {resume},
  pdfpagemode = UseNone
}

\geometry{
  %body={6.5in, 8.5in},
  left=0.25in,
  right=0.25in,
  top=0.15in,
  bottom=0.25in
}

% Customize page headers
\pagestyle{myheadings}
\markright{\name}
\thispagestyle{empty}

% Custom section fonts
\usepackage{sectsty}
\sectionfont{\rmfamily\mdseries\Large\bf\sectionrule{0ex}{0pt}{-1ex}{1pt}}
\subsectionfont{\rmfamily\mdseries\bf\normalsize}
\subsubsectionfont{\rmfamily\mdseries\itshape\normalsize}

% Other possible font commands include:
% \ttfamily for teletype,
% \sffamily for sans serif,
% \bfseries for bold,
% \scshape for small caps,
% \normalsize, \large, \Large, \LARGE sizes.

% Don't indent paragraphs.
\setlength\parindent{0em}
%Control spacing between lines and after paragraphs
\setlength{\parskip}{0em}
\renewcommand{\baselinestretch}{0}

% Section spacings
\usepackage{titlesec}
\titlespacing*{\section}{0em}{0em}{0em}
\titlespacing*{\subsection}{0em}{0em}{0em}
\titlespacing*{\subsubsection}{0em}{0em}{0em}


% Control lists
\renewenvironment{itemize}{
  \begin{list}{$\bullet$}{
    \setlength{\itemsep}{0em}
    \setlength{\parskip}{0em}
    \setlength{\parsep}{0em} 
    \setlength{\topsep}{0em} 
  }
}{
  \end{list}
}

\begin{document}

% Place name at left
%{\huge \name}

% Alternatively, print name centered:
\centerline{\LARGE \name}

\begin{center}

\small{(580)641-3030 $\bullet$ josiah.couch@bc.edu $\bullet$ https://www.linkedin.com/in/josiah-couch/}

\end{center}

%\section*{Summary}

%\begin{itemize}

%\item 6 published papers, 330 citations, h-index of 6, Erd\"os number $\leq 3$

%\item Experience doing original research in a collaborative environment

%\item Advanced skills in mathematics and quantitative problem-solving

%\item Experience using Python and C/C++ to analyze data

%\end{itemize}

\section*{Education}

\subsection*{University of Texas at Austin, Austin, TX}
\subsubsection*{Ph.D in Physics (3.6 GPA) \hfill August 2013 - May 2021}

\begin{itemize}
\item Adviser: Willy Fischler
\item Field: AdS/CFT
\end{itemize}

\subsection*{Oklahoma State University, Stillwater, Oklahoma}
\subsubsection*{B.S in Physics and B.S. in Mathematics \hfill August 2009 - August 2013}

\section*{Experience}

\subsection*{Boston College, Chestnut Hill, MA} 
\subsubsection*{Postdoctoral Research Fellow in Computer Science Department\hfill January 2021 - Present}
    \begin{itemize}
        
        \item Working on problems related to the graph alignment or graph matching problem, an example of a problem in combinatorial optimization.
        
        \item Applying statistical techniques to determine failure conditions for the maximum a posteriori (MAP)  estimator.
        
    \end{itemize}

\subsection*{University of Texas at Austin, Austin, TX} 
\subsubsection*{Ph.D. Candidate in Weinberg Theory Group under Prof. Willy Fischler \hfill August 2013 - May 2021}
    \begin{itemize}

        \item Published six papers in peer-reviewed journals with a total of 330 citations by 240 unique works (according to inspirehep.net). 
        \item Collaborated with 14 other researchers, including six external collaborators from two institutions.
        \item Gave six external talks, five at conferences, and participated in three additional conferences and schools
        \item Awarded 2018 OGS Summer Only Fellowship
        \item Solved complex problems using differential geometry, linear algebra, and differential equations, along with the Mathematica software package
        
    \end{itemize}
    
\subsubsection*{Teaching Assistant \hfill August 2013 - December 2020}
    \begin{itemize}
    
        \item Evaluated student work and provided critical feedback
        \item Modeled analytic problem solving to students through official solutions to problem sets and in live review sessions
        \item Coached students in quantitative problem solving as well as understanding of course subject matters
        \item Courses included graduate quantum mechanics (1 semester), undergraduate quantum mechanics (6 semesters), forensic science (2 semesters), electricity and magnetism (3 semesters), and engineering physics labs (5 semesters)
        %\item Subjects include engineering physics, forensic science, electricity, and magnetism at the undergraduate level, as well as quantum mechanics at both the undergraduate and graduate level
        \item Nominated for the UT Services for Students with Disabilities Appreciation Award in Fall 2016
    \end{itemize}
    
\subsubsection*{Graduate Research Assistant with Center for Particles and Fields under Prof. Peter Onyisi \hfill June - July 2014}
    \begin{itemize}
        
        \item Work related to the ATLAS experiment at the Large Hadron Collider
        \item Optimized feature selection for a particle physics analysis using simulated data, python, and the ROOT TMVA boosted decision tree implementation (used through pyROOT)
        \item Developed a python script to pass configuration file from database to fast tracker system
        
    \end{itemize}

\subsection*{Oklahoma State University, Stillwater, OK}
\subsubsection*{Undergraduate Researcher in Experimental Particle Physics Group under Prof. Flera Rizatdinova \hfill January 2010 - May 2012}
    \begin{itemize}
        \item Analyzed particle physics data from the DØ and ATLAS experiments using C/C++ and the ROOT package
    \end{itemize}

\section*{Skills}

\begin{itemize}

\item \textbf{Confident with}: Python, Pandas, Mathematica, LaTex%, Quantitative Problem Solving, Research, Collaboration, Linear Algebra, Differential Equations, Quantum Mechanics, General Relativity

\item \textbf{Exposure to}: Sci-Kit Learn, Git, Java, Javascript, Haskell, C/C++, HTML, SQL%, ROOT, Functional Analysis, Algebraic Topology, Quantum Field Theory

\end{itemize}

%\section*{Interests}

%Quantum Gravity, Quantum Computing, Programming, Machine Learning, Photography, Mathematics, Languages

\end{document}

